\documentclass[11pt,a4paper]{report}

% 패키지 불러오기
\usepackage{kotex}           % 한글
\newfontfamily\hanjafont{Noto Serif CJK KR}  % 예: 윈도우 내장 중국어 폰트
\usepackage[a4paper,
  left=30mm,
  right=25mm,
  top=30mm,
  bottom=30mm,
  includeheadfoot]{geometry}
  % 여백
\usepackage{setspace}           % 줄 간격
\usepackage{hyperref}


\usepackage{lmodern}
\usepackage{etoc}
\usepackage{xparse}
\usepackage{expl3}

\usepackage{fancyhdr}           % 머리말/꼬리말
\usepackage{graphicx}           % 이미지
\usepackage{amsmath,amssymb}    % 수학
\usepackage{biblatex}           % 참고문헌
\addbibresource{bibliography.bib}  % 참고문헌 bib 파일

% 폰트 설정 (XeLaTeX인 경우)
\usepackage{fontspec}
\setmainfont{나눔명조OTF}[
  BoldFont = 나눔명조OTF Bold]

\renewcommand{\contentsname}{}

\makeatletter
\spaceskip=0.7em plus 0.2em minus 0.2em
\makeatother




  % 들여쓰기 없음
\setlength{\parskip}{6pt}       % 문단 간격
\linespread{1.5}            % 1.5줄 간격

% 본문 시작
\begin{document}

\pagenumbering{gobble}          % 페이지 번호 숨김

\begin{titlepage}
    \begin{center}

        {\fontsize{14pt}{\baselineskip}\selectfont \textbf{농업자원경제학 석사 학위논문}}\\
        \vspace{4.3cm}

        {\fontsize{22pt}{\baselineskip}\selectfont \textbf{공적개발원조(ODA)가}}\\
        \vspace{.6cm}
        {\fontsize{22pt}{\baselineskip}\selectfont \textbf{공여국에 미치는 경제·사회문화적 효과}}\\
        \vspace{.6cm}

        {\fontsize{16pt}{\baselineskip}\selectfont \textbf{- 한국의 {\hanjafont 對} 중점협력국 ODA를 중심으로 -}}\\
        \vspace{5.8cm}

        {\fontsize{14pt}{\baselineskip}\selectfont \textbf{2025년 2월}}\\
        \vspace{3.5cm}

        
        {\fontsize{16pt}{\baselineskip}\selectfont \textbf{서울대학교 농업생명과학대학}}\\
        \vspace{.4cm}
        {\fontsize{14pt}{\baselineskip}\selectfont \textbf{행정학과 행정학 전공}}\\
        \vspace{.4cm}


        \makebox[3cm][s]{{\fontsize{16pt}{\baselineskip}\selectfont \textbf{박지민}}}\\

    \end{center}
\end{titlepage}

\begin{titlepage}
    \begin{center}

        {\fontsize{22pt}{\baselineskip}\selectfont \textbf{공적개발원조(ODA)가}}\\
        \vspace{.6cm}
        {\fontsize{22pt}{\baselineskip}\selectfont \textbf{공여국에 미치는 경제·사회문화적 효과}}\\
        \vspace{.6cm}

        {\fontsize{16pt}{\baselineskip}\selectfont \textbf{- 한국의 {\hanjafont 對} 중점협력국 ODA를 중심으로 -}}\\
        \vspace{1.5cm}

        \makebox[3cm][s]{{\fontsize{16pt}{\baselineskip}\selectfont \textbf{지도교수\ }}}
            \makebox[3cm][s]{{\fontsize{16pt}{\baselineskip}\selectfont \textbf{이석현}}}\\
        \vspace{1.5cm}

        {\fontsize{16pt}{\baselineskip}\selectfont \textbf{이 논문을 농업자원경제학 석사 학위논문으로 제출함}}\\
        \vspace{0.2cm}

        {\fontsize{14pt}{\baselineskip}\selectfont \textbf{2025년 2월}}\\
        \vspace{1cm}

        {\fontsize{16pt}{\baselineskip}\selectfont \textbf{서울대학교 농업생명과학대학}}\\
        \vspace{.2cm}
        {\fontsize{14pt}{\baselineskip}\selectfont \textbf{농경제사회학과 농업자원경제학 전공}}\\
        \vspace{.2cm}
        \makebox[3cm][s]{{\fontsize{16pt}{\baselineskip}\selectfont \textbf{박지민}}}\\
        \vspace{1cm}

        {\fontsize{16pt}{\baselineskip}\selectfont \textbf{박지민의 석사 학위논문을 인준함}}\\
        \vspace{0.2cm}

        {\fontsize{14pt}{\baselineskip}\selectfont \textbf{2025년 2월}}\\
        \vspace{2.5cm}
        
        \makebox[3cm][s]{{\fontsize{16pt}{\baselineskip}\selectfont \textbf{위원장}}}
            \underline{\makebox[7cm][s]{{\fontsize{16pt}{\baselineskip}\selectfont \textbf{\ 전소희\ (인)}}}}\\[2em]
        \makebox[3cm][s]{{\fontsize{16pt}{\baselineskip}\selectfont \textbf{부위원장}}}
            \underline{\makebox[7cm][s]{{\fontsize{16pt}{\baselineskip}\selectfont \textbf{\ 전소희\ (인)}}}}\\[2em]
        \makebox[3cm][s]{{\fontsize{16pt}{\baselineskip}\selectfont \textbf{위원}}}
            \underline{\makebox[7cm][s]{{\fontsize{16pt}{\baselineskip}\selectfont \textbf{\ 전소희\ (인)}}}}\\


    \end{center}
\end{titlepage}






\pagenumbering{roman}           % 목차 등은 소문자 로마자

\begin{center}
    {\LARGE \bfseries 국문초록}
\end{center}


\vspace{1cm}

본 연구는 대한민국의 농업자원경제학 분야에서의 구조적 변화와 정책 대응을 분석한 것이다.
기존 문헌 검토를 통해 주요 이슈를 정리하고, 실증 분석을 통해 최근 10년간의 농업정책 변화가 
자원 배분과 농가소득에 미친 영향을 평가하였다. 주요 결과는 다음과 같다. 첫째, \ldots\ 
둘째, \ldots\ 이러한 결과는 향후 농업정책의 설계에 있어 실질적 함의를 제공할 수 있다.


\noindent
\makebox[1.1cm][s]{{\textbf{주요어}}}
\textbf{:}
\textbf{\abstractkeyword}\\
\makebox[1.1cm][s]{{\textbf{학번}}}
\textbf{:}
\textbf{\studentnumber}\\


\begin{center}
  {\fontsize{14pt}{\baselineskip}\selectfont \textbf{목차}}\\
  \vspace{1cm}
  \nameref{chap:chapter1} \dotfill \pageref{chap:chapter1} \par
\hspace{1em}\nameref{sec:chapter1_1} \dotfill \pageref{sec:chapter1_1} \par
\hspace{1em}\nameref{sec:chapter1_2} \dotfill \pageref{sec:chapter1_2} \par
\hspace{2em}\nameref{subsec:chapter1_2_1} \dotfill \pageref{subsec:chapter1_2_1} \par
\hspace{2em}\nameref{subsec:chapter1_2_2} \dotfill \pageref{subsec:chapter1_2_2} \par
\nameref{chap:chapter2} \dotfill \pageref{chap:chapter2} \par
\hspace{1em}\nameref{sec:chapter2_1} \dotfill \pageref{sec:chapter2_1} \par
\hspace{1em}\nameref{sec:chapter2_2} \dotfill \pageref{sec:chapter2_2} \par
\hspace{2em}\nameref{subsec:chapter2_2_1} \dotfill \pageref{subsec:chapter2_2_1} \par
\hspace{2em}\nameref{subsec:chapter2_2_2} \dotfill \pageref{subsec:chapter2_2_2} \par

\end{center}




\newpage

\pagenumbering{arabic}          % 본문은 숫자 페이지
\setcounter{page}{1}


\chapter{제1장 서론}



\section{제 1절 연구 배경}
한국 농업은 기후 변화와 글로벌 시장의 영향으로 다양한 도전에 직면하고 있다.

\section{제 2절 연구 목적}
본 연구는 농업자원경제학 분야에서 정책 변화가 농가 소득에 미친 영향을 분석하고자 한다.

\subsection{1. 세부 목적 1}
정책 변화 전후의 구조 비교

\subsection{2. 세부 목적 2}
실증 분석을 통한 영향 추정

\chapter{제2장 서론}


\section{제 1절 연dd 배경}
한국 농업은 기후 변화와 글로벌 시장의 영향으로 다양한 도전에 직면하고 있다.

\section{제 2절 aaa 목적}
본 연구는 농업자원경제학 분야에서 정책 변화가 농가 소득에 미친 영향을 분석하고자 한다.

\subsection{1. 세부 dfsf적 1}
정책 변화 전후의 구조 비교

\subsection{2. 세부 dfhhjj 2}
실증 분석을 통한 영향 추정
% 추가 장은 자유롭게 추가


            % 부록 (선택)

\printbibliography              % 참고문헌 출력

\end{document}
