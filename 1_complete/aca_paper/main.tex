\documentclass[11pt,a4paper]{report}

% 패키지 불러오기
\usepackage{kotex}           % 한글
\newfontfamily\hanjafont{Noto Serif CJK KR}  % 예: 윈도우 내장 중국어 폰트
\usepackage[a4paper,
  left=30mm,
  right=25mm,
  top=30mm,
  bottom=30mm,
  includeheadfoot]{geometry}
  % 여백
\usepackage{setspace}     


\usepackage{lmodern}
\usepackage{xparse}
\usepackage{expl3}

\usepackage{fancyhdr}           % 머리말/꼬리말
\usepackage{graphicx}           % 이미지
\usepackage{amsmath,amssymb}    % 수학


\usepackage[
  backend=biber
  ,style=authoryear
  ,language=korean
  ]{biblatex}


\addbibresource{references.bib} % .bib 파일 연결


% 폰트 설정 (XeLaTeX인 경우)
\usepackage{fontspec}
\setmainfont{나눔명조OTF}[
  BoldFont = 나눔명조OTF Bold]


% \renewcommand{\contentsname}{} %  빈칸으로 (아예 안 보이게)
\renewcommand{\chaptername}{제~\thechapter~장} % "제 1 장" 처럼 출력


\usepackage{titlesec}

% 챕터 제목 크기 및 스타일 변경
\titleformat{\chapter}[hang] % display: 별도 줄에 크게
  {\normalfont\Large\bfseries\centering}   % 글꼴: 기본, Huge 크기, 굵게
  {\chaptername }   % 번호 앞에 'Chapter' 붙이기 (영어 기준)
  {20pt}                        % 번호와 제목 사이 간격
  {}                       % 제목 글꼴 (추가 설정)

% 섹션 제목 크기 및 스타일 변경
\renewcommand{\thesection}{\arabic{section}}

\titleformat{\section}
  {\normalfont\Large\bfseries}      % 글꼴
  {제~\thesection~절}                    % 번호 스타일 (여기서 '1.'처럼 점 붙임)
  {1em}                            % 번호와 제목 사이 간격
  {}


% 서브섹션 제목 크기 및 스타일 변경
\renewcommand{\thesubsection}{\arabic{subsection}}

\titleformat{\subsection}
  {\normalfont\large\bfseries} % 기본 크기, 굵게
  {\thesubsection .}
  {1em}
  {}


  

\renewcommand{\contentsname}{목 차}

% \renewcommand{\figurename}{그림}
% \renewcommand{\tablename}{표}
\renewcommand{\listfigurename}{그림 목차}
\renewcommand{\listtablename}{표 목차}



\usepackage{caption}

\DeclareCaptionLabelFormat{brackets}{[#1~#2]}

\captionsetup{
  labelformat=brackets,
  labelsep=space,
  format=hang,
  font=normalsize
}

\captionsetup[figure]{name=그림}
\captionsetup[table]{name=표}



\makeatletter
\spaceskip=0.7em plus 0.2em minus 0.2em
\makeatother


\titlespacing*{\chapter}{0pt}{0pt}{80pt}


  % 들여쓰기 없음
\setlength{\parskip}{0pt}       % 문단 간격
\linespread{1.5}            % 1.5줄 간격


\usepackage{tocloft}

\renewcommand{\cftchapfont}{\Large\bfseries} % 목차에서 chapter 글씨 크기, 굵게
\renewcommand{\cftsecfont}{\large}            % 목차에서 section 글씨 크기
\renewcommand{\cftsubsecfont}{\normalsize}    % 목차에서 subsection 글씨 크기


\renewcommand{\cftchappagefont}{\Large}  % chapter 페이지 번호 크기
\renewcommand{\cftsecpagefont}{\large}   % section 페이지 번호 크기
\renewcommand{\cftsubsecpagefont}{\normalsize} % subsection 페이지 번호 크


% 목차 항목 간 행간 간격 조절
\setlength{\cftbeforechapskip}{10pt}   % 챕터 항목 위 간격 (행간)
\setlength{\cftbeforesecskip}{5pt}     % 섹션 항목 위 간격
\setlength{\cftbeforesubsecskip}{0pt}  % 서브섹션 항목 위 간격




% \makeatletter
% \patchcmd{\@makechapterhead}{50\p@}{0pt}{}{}
% \patchcmd{\@makeschapterhead}{50\p@}{0pt}{}{}
% \makeatother






\usepackage{etoolbox}

\usepackage{etoc}

% 목차 제목 글꼴/크기 변경 예시 (글꼴 크기와 굵기 조정)
\usepackage{etoolbox}
\makeatletter
\patchcmd{\tableofcontents}
  {\chapter*{\contentsname}}% 원래 코드
  {\chapter*{\fontsize{20}{24}\selectfont\bfseries\centering 목차}}% 중앙정렬 추가
  {}{}
\patchcmd{\listoffigures}
  {\chapter*{\listfigurename}}
  {\chapter*{\fontsize{18}{22}\selectfont\bfseries\centering 그림 목록}}% 중앙정렬 추가
  {}{}
\patchcmd{\listoftables}
  {\chapter*{\listtablename}}
  {\chapter*{\fontsize{18}{22}\selectfont\bfseries\centering 표 목록}}% 중앙정렬 추가
  {}{}
\makeatother





% 본문 시작
%%%%%%%%%%%%%%%%%%%%%%%%%%%%%%%%%%%%%%%%%%%%%%%%%%%%%%%%%%

\begin{document}

\pagenumbering{gobble}          % 페이지 번호 숨김

\begin{titlepage}
    \begin{center}

        {\Large \textbf{\major \, \course \, 학위논문}}\\
        \vspace{1.5cm}


        \begin{spacing}{2.7}
            {\Huge \textbf{\maintitle}}\\
        \end{spacing}

        {\LARGE \textbf{- \subtitle -}}\\
        \vspace{5.8cm}

        {\Large \textbf{\publishyearmonth}}\\
        \vspace{3.3cm}

        
        {\LARGE \textbf{\univ \, \affiliation}}\\
        \vspace{.4cm}
        {\Large \textbf{\department \, \major \, 전공}}\\
        \vspace{.4cm}


        \makebox[2.5cm][s]{{\LARGE \textbf{\authorname}}}\\

    \end{center}
\end{titlepage}

\begin{titlepage}
    \begin{center}

        \begin{spacing}{2.7}
            {\Huge \textbf{\maintitle}}\\
        \end{spacing}

        {\LARGE \textbf{- \subtitle -}}\\
        \vspace{1.5cm}

        \makebox[3cm][s]{{\LARGE \textbf{지도교수 \,}}}
            \makebox[2.5cm][s]{{\LARGE \textbf{\supervisor}}}\\
        \vspace{1.5cm}

        {\LARGE \textbf{이 논문을 \major \, \course \, 학위논문으로 제출함}}\\
        \vspace{0.2cm}

        {\Large \textbf{\submityearmonth}}\\
        \vspace{1cm}

        {\LARGE \textbf{\univ \, \affiliation}}\\
        \vspace{.2cm}
        {\Large \textbf{\department \, \major 전공}}\\
        \vspace{.2cm}
        \makebox[2.5cm][s]{{\LARGE \textbf{\authorname}}}\\
        \vspace{1cm}

        {\LARGE \textbf{\authorname 의\, \course \, 학위논문을 인준함}}\\
        \vspace{0.2cm}

        {\Large \textbf{\publishyearmonth}}\\
        \vspace{2.5cm}
        
        \makebox[3cm][s]{{\LARGE \textbf{위원장}}}
            \underline{\makebox[7cm][s]{{\LARGE \textbf{\, \comchair \, (인)}}}}\\[2em]
        \makebox[3cm][s]{{\LARGE \textbf{부위원장}}}
            \underline{\makebox[7cm][s]{{\LARGE \textbf{\, \comvicechair \, (인)}}}}\\[2em]
        \makebox[3cm][s]{{\LARGE \textbf{위원}}}
            \underline{\makebox[7cm][s]{{\LARGE \textbf{\, \commember \, (인)}}}}\\


    \end{center}
\end{titlepage}




\pagenumbering{roman}           % 목차 등은 소문자 로마자

% 국문 초록 제목 설정
\begin{center}
\section*{\makebox[2.5cm][s]{{\fontsize{16pt}{\baselineskip}\selectfont \textbf{국문초록}}}}
\end{center}

\vspace{1cm}

본 연구는 대한민국의 농업자원경제학 분야에서의 구조적 변화와 정책 대응을 분석한 것이다.
기존 문헌 검토를 통해 주요 이슈를 정리하고, 실증 분석을 통해 최근 10년간의 농업정책 변화가 
자원 배분과 농가소득에 미친 영향을 평가하였다. 주요 결과는 다음과 같다. 첫째, \ldots\ 
둘째, \ldots\ 이러한 결과는 향후 농업정책의 설계에 있어 실질적 함의를 제공할 수 있다.


\noindent
\makebox[1.1cm][s]{{\textbf{주요어}}}
\textbf{:}
\textbf{전소희}\\
\makebox[1.1cm][s]{{\textbf{학번}}}
\textbf{:}
\textbf{2025-20181}\\




\tableofcontents

\newpage

\listoffigures
\listoftables


\newpage

\pagenumbering{arabic}          % 본문은 숫자 페이지
\setcounter{page}{1}


\chapter{제1장 서론}



\section{제 1절 연구 배경}
한국 농업은 기후 변화와 글로벌 시장의 영향으로 다양한 도전에 직면하고 있다.

\section{제 2절 연구 목적}
본 연구는 농업자원경제학 분야에서 정책 변화가 농가 소득에 미친 영향을 분석하고자 한다.

\subsection{1. 세부 목적 1}
정책 변화 전후의 구조 비교

\subsection{2. 세부 목적 2}
실증 분석을 통한 영향 추정
\chapter{제2장 이론적 배경 및 선행연구}

본 장에서는 농업자원경제학의 이론적 토대를 구성하는 주요 이론들을 정리하고,  
국내외에서 진행된 관련 선행연구들을 검토함으로써 본 연구의 차별성과 의의를 도출한다.  
특히 자원 배분 이론, 정책 충격 분석, 농가 행태 이론 등을 중심으로 이론적 틀을 구축하고자 한다.

---

\section{제 1절 이론적 배경}

농업자원경제학은 제한된 자원을 효율적으로 활용하여 최대한의 산출을 도출하는 것을 목표로 하는 경제학의 한 분야이다.  
이론적으로는 미시경제학적 자원 배분 이론, 비용-편익 분석, 생산함수 모형 등을 기반으로 농업 내 의사결정을 분석한다.

농가의 자원 배분 결정은 토지, 노동, 자본, 기술 등 생산요소의 상대적 희소성과 수익성에 따라 달라지며,  
정책(보조금, 세제 혜택, 규제 등)은 이러한 결정에 중요한 외생적 영향을 미친다.

% 그림: 분석 프레임워크 도식 (빈 공간 확보)
\begin{figure}[htbp]
  \centering
  \rule{0.8\linewidth}{5cm} % 그림 공간 확보
  \caption{농업자원경제학의 분석 프레임워크}\label{fig:chapter2_1}
\end{figure}

---

\section{제 2절 선행연구 고찰}

국내외 선행연구들은 농업정책이 농가소득, 자원배분, 환경성과 등에 미치는 영향을 다각도로 분석해왔다.  
특히 선진국의 경우 보조금 정책과 환경규제, 청년농 유입정책 등과 관련한 정량 분석이 활발히 수행되었다.  

반면 국내 연구는 제도 변화의 전후 비교, 설문기반 분석에 집중되어 있어 실증적 인과 추정에는 한계가 있었다.  
본 연구는 이러한 한계를 극복하고자 패널 데이터 기반의 분석 방법론을 활용하며,  
정책 효과의 실제 규모를 추정하는 데 중점을 둔다.

% 표: 주요 선행연구 비교
% \begin{table}[htbp]
%   \centering
%   \begin{tabular}{|c|c|c|c|}
%     \hline
%     연구자 & 대상국 & 분석 방법 & 주요 변수 \\
%     \hline
%     Kim (2018) & 한국 & OLS, 고정효과 & 농가소득, 정책참여 \\
%     Lee et al. (2020) & 미국 & DID 분석 & 농업 생산성, 정부보조 \\
%     Zhang (2019) & 중국 & 패널회귀 & 경지면적, 자본투입 \\
%     \hline
%   \end{tabular}
%   \caption{주요 선행연구 비교}\label{tab:chapter2_1}
% \end{table}

---

\subsection{1. 농가 행태 이론 기반 분석}

농가의 생산 결정은 위험 회피 성향, 가격 예측, 기술 수용성 등의 내생적 요인과  
정책 제도, 시장 가격, 기후 등 외생적 요인에 의해 복합적으로 영향을 받는다.

이 절에서는 농가 행태 이론을 바탕으로 농가가 어떻게 자원을 배분하고,  
정책 변화에 어떤 반응을 보이는지를 수리적 모형으로 해석한다.

% 그림: 농가 의사결정 흐름도 (빈 공간 확보)
\begin{figure}[htbp]
  \centering
  \rule{0.75\linewidth}{5cm}
  \caption{농가의 자원 배분 및 정책 반응 구조도}\label{fig:chapter2_2}
\end{figure}

---

\subsection{2. 정책 효과 분석모형 소개}

정책 효과를 정량적으로 추정하기 위해 본 연구는 다음과 같은 분석모형을 활용한다.

1. \textbf{DID (Difference-in-Differences)}: 정책 개입 전후 변화량을 비교  
2. \textbf{고정효과 패널 회귀 (Fixed Effects)}: 시간불변 요인 통제  
3. \textbf{도구변수법 (IV)}: 내생성 문제 보정

이 모형들을 통해 정책 도입이 농가의 수익성, 자원배분, 구조 변화에 어떤 영향을 미쳤는지를 추정하고,  
그 결과에 따라 정책의 타당성과 개선 방향을 제시한다.

% 표: 분석모형 요약
% \begin{table}[htbp]
%   \centering
%   \begin{tabular}{|l|c|l|}
%     \hline
%     모형 & 주요 특징 & 적용 목적 \\
%     \hline
%     DID & 전후 비교, 외생충격 측정 & 정책 개입 효과 \\
%     패널 회귀 & 개별 농가 고정효과 통제 & 장기 추세 분석 \\
%     IV 회귀 & 내생성 문제 해결 & 보조금 효과의 인과 추정 \\
%     \hline
%   \end{tabular}
%   \caption{본 연구에서 사용된 정책 효과 분석모형}\label{tab:chapter2_2}
% \end{table}

---

\subsection{요약}

이 장에서는 농업자원경제학의 이론적 기초와 농가의 의사결정 과정을 개괄하고,  
국내외 선행연구를 종합 분석하였다. 이어지는 장에서는 자료 및 분석방법을 구체적으로 설명한다.

% 추가 장은 자유롭게 추가


\printbibliography[title={참 고 문 헌}]


\begin{center}
  \begin{spacing}{1.5} % 2.5줄 간격 (2보다 넓음)

    {\LARGE \bfseries Abstract} \par
    {\Huge \bfseries \maintitleen}

  \end{spacing}
\end{center}

\begin{flushright}
    \begin{spacing}{2}
        {\Large \authornameen} \par
        {\Large \affiliationen} \par
        {\Large \univen} \par
     \end{spacing}
\end{flushright}



The purpose of this study is to analyze the economic and cultural
effect of Official Development Assistance(ODA) of Donor Country.
Since joining the OECD Development Assistance Committee (DAC) in
November 2009, Korea has been strengthening its responsibilities and
obligations as a donor country. The scale of Official Development
Assistance (ODA) has more than doubled over the past decade,
increasing from $1.2 billion in 2010 to approximately $3.1 billion in
2023, ranking 14th among all 31 DAC member countries.
ODA fundamentally aims at humanitarian goals, focusing on
poverty alleviation and economic development in the international

\noindent
\makebox[3cm][l]{{\textbf{Keywords}}}
\textbf{:}
\textbf{\abstractkeyworden}\\
\makebox[3cm][l]{{\textbf{Student Number}}}
\textbf{:}
\textbf{\studentnumber}\\


\end{document}
