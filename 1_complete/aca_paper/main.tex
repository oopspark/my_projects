\documentclass[11pt,a4paper]{report}

% 패키지 불러오기
\usepackage{kotex}           % 한글
\newfontfamily\hanjafont{Noto Serif CJK KR}  % 예: 윈도우 내장 중국어 폰트
\usepackage[a4paper,
  left=30mm,
  right=25mm,
  top=30mm,
  bottom=30mm,
  includeheadfoot]{geometry}

\usepackage{multirow}
\usepackage{hhline}
\usepackage{array}
\usepackage{booktabs} 

\usepackage{makecell}
\usepackage[table]{xcolor} % 색상 패키지



\usepackage{tabularx}
% \usepackage{ltablex} % tabularx의 확장판
% \keepXColumns % 중복 방지

% 열 정의: X열을 포함해서 자동 너비 분배 + 회색 배경 열 정의
% \newcolumntype{C}{>{\columncolor{gray!20}}c}  % 고정폭 회색열
\newcolumntype{Y}{>{\centering \arraybackslash}X}  % 가변폭 열 (tabularx 전용)



\usepackage{pgfkeys}
\usepackage{csvsimple}

  % 여백
\usepackage{setspace}     


\usepackage{lmodern}
\usepackage{xparse}
\usepackage{expl3}

\usepackage{fancyhdr}           % 머리말/꼬리말
\usepackage{graphicx}           % 이미지
\usepackage{amsmath,amssymb}    % 수학


\usepackage[
  backend=biber
  ,style=authoryear
  ,language=korean
  ]{biblatex}


\addbibresource{data/references.bib} % .bib 파일 연결


% 폰트 설정 (XeLaTeX인 경우)
\usepackage{fontspec}
\setmainfont{나눔명조OTF}[
  BoldFont = 나눔명조OTF Bold]


% \renewcommand{\contentsname}{} %  빈칸으로 (아예 안 보이게)
\renewcommand{\chaptername}{제~\thechapter~장} % "제 1 장" 처럼 출력


\usepackage{titlesec}

% 챕터 제목 크기 및 스타일 변경
\titleformat{\chapter}[hang] % display: 별도 줄에 크게
  {\normalfont\Large\bfseries\centering}   % 글꼴: 기본, Huge 크기, 굵게
  {\chaptername }   % 번호 앞에 'Chapter' 붙이기 (영어 기준)
  {20pt}                        % 번호와 제목 사이 간격
  {}                       % 제목 글꼴 (추가 설정)

% 섹션 제목 크기 및 스타일 변경
\renewcommand{\thesection}{\arabic{section}}

\titleformat{\section}
  {\normalfont\Large\bfseries}      % 글꼴
  {제~\thesection~절}                    % 번호 스타일 (여기서 '1.'처럼 점 붙임)
  {1em}                            % 번호와 제목 사이 간격
  {}


% 서브섹션 제목 크기 및 스타일 변경
\renewcommand{\thesubsection}{\arabic{subsection}}

\titleformat{\subsection}
  {\normalfont\large\bfseries} % 기본 크기, 굵게
  {\thesubsection .}
  {1em}
  {}




\renewcommand{\contentsname}{목 차}

% \renewcommand{\figurename}{그림}
% \renewcommand{\tablename}{표}
\renewcommand{\listfigurename}{그림 목차}
\renewcommand{\listtablename}{표 목차}



\usepackage{caption}

\DeclareCaptionLabelFormat{brackets}{[#1~#2]}

\captionsetup{
  labelformat=brackets,
  labelsep=space,
  format=hang,
  font=normalsize
}

\captionsetup[figure]{name=그림}
\captionsetup[table]{name=표}



\makeatletter
\spaceskip=0.7em plus 0.2em minus 0.2em
\makeatother


\titlespacing*{\chapter}{0pt}{0pt}{80pt}


  % 들여쓰기 없음
\setlength{\parskip}{0pt}       % 문단 간격
\linespread{1.5}            % 1.5줄 간격


\usepackage{tocloft}

\renewcommand{\cftchapfont}{\Large\bfseries} % 목차에서 chapter 글씨 크기, 굵게
\renewcommand{\cftsecfont}{\large}            % 목차에서 section 글씨 크기
\renewcommand{\cftsubsecfont}{\normalsize}    % 목차에서 subsection 글씨 크기


\renewcommand{\cftchappagefont}{\Large}  % chapter 페이지 번호 크기
\renewcommand{\cftsecpagefont}{\large}   % section 페이지 번호 크기
\renewcommand{\cftsubsecpagefont}{\normalsize} % subsection 페이지 번호 크


% 목차 항목 간 행간 간격 조절
\setlength{\cftbeforechapskip}{10pt}   % 챕터 항목 위 간격 (행간)
\setlength{\cftbeforesecskip}{5pt}     % 섹션 항목 위 간격
\setlength{\cftbeforesubsecskip}{0pt}  % 서브섹션 항목 위 간격




% \makeatletter
% \patchcmd{\@makechapterhead}{50\p@}{0pt}{}{}
% \patchcmd{\@makeschapterhead}{50\p@}{0pt}{}{}
% \makeatother






\usepackage{etoolbox}

\usepackage{etoc}

% 목차 제목 글꼴/크기 변경 예시 (글꼴 크기와 굵기 조정)
\usepackage{etoolbox}
\makeatletter
\patchcmd{\tableofcontents}
  {\chapter*{\contentsname}}% 원래 코드
  {\chapter*{\fontsize{20}{24}\selectfont\bfseries\centering 목차}}% 중앙정렬 추가
  {}{}
\patchcmd{\listoffigures}
  {\chapter*{\listfigurename}}
  {\chapter*{\fontsize{18}{22}\selectfont\bfseries\centering 그림 목록}}% 중앙정렬 추가
  {}{}
\patchcmd{\listoftables}
  {\chapter*{\listtablename}}
  {\chapter*{\fontsize{18}{22}\selectfont\bfseries\centering 표 목록}}% 중앙정렬 추가
  {}{}
\makeatother





% 본문 시작
%%%%%%%%%%%%%%%%%%%%%%%%%%%%%%%%%%%%%%%%%%%%%%%%%%%%%%%%%%

\begin{document}

\pagenumbering{gobble}          % 페이지 번호 숨김

\begin{titlepage}
    \begin{center}

        {\fontsize{14pt}{\baselineskip}\selectfont \textbf{농업자원경제학 석사 학위논문}}\\
        \vspace{4.3cm}

        {\fontsize{22pt}{\baselineskip}\selectfont \textbf{공적개발원조(ODA)가}}\\
        \vspace{.6cm}
        {\fontsize{22pt}{\baselineskip}\selectfont \textbf{공여국에 미치는 경제·사회문화적 효과}}\\
        \vspace{.6cm}

        {\fontsize{16pt}{\baselineskip}\selectfont \textbf{- 한국의 {\hanjafont 對} 중점협력국 ODA를 중심으로 -}}\\
        \vspace{5.8cm}

        {\fontsize{14pt}{\baselineskip}\selectfont \textbf{2025년 2월}}\\
        \vspace{3.5cm}

        
        {\fontsize{16pt}{\baselineskip}\selectfont \textbf{서울대학교 농업생명과학대학}}\\
        \vspace{.4cm}
        {\fontsize{14pt}{\baselineskip}\selectfont \textbf{행정학과 행정학 전공}}\\
        \vspace{.4cm}


        \makebox[3cm][s]{{\fontsize{16pt}{\baselineskip}\selectfont \textbf{박지민}}}\\

    \end{center}
\end{titlepage}

\begin{titlepage}
    \begin{center}

        {\fontsize{22pt}{\baselineskip}\selectfont \textbf{공적개발원조(ODA)가}}\\
        \vspace{.6cm}
        {\fontsize{22pt}{\baselineskip}\selectfont \textbf{공여국에 미치는 경제·사회문화적 효과}}\\
        \vspace{.6cm}

        {\fontsize{16pt}{\baselineskip}\selectfont \textbf{- 한국의 {\hanjafont 對} 중점협력국 ODA를 중심으로 -}}\\
        \vspace{1.5cm}

        \makebox[3cm][s]{{\fontsize{16pt}{\baselineskip}\selectfont \textbf{지도교수\ }}}
            \makebox[3cm][s]{{\fontsize{16pt}{\baselineskip}\selectfont \textbf{이석현}}}\\
        \vspace{1.5cm}

        {\fontsize{16pt}{\baselineskip}\selectfont \textbf{이 논문을 농업자원경제학 석사 학위논문으로 제출함}}\\
        \vspace{0.2cm}

        {\fontsize{14pt}{\baselineskip}\selectfont \textbf{2025년 2월}}\\
        \vspace{1cm}

        {\fontsize{16pt}{\baselineskip}\selectfont \textbf{서울대학교 농업생명과학대학}}\\
        \vspace{.2cm}
        {\fontsize{14pt}{\baselineskip}\selectfont \textbf{농경제사회학과 농업자원경제학 전공}}\\
        \vspace{.2cm}
        \makebox[3cm][s]{{\fontsize{16pt}{\baselineskip}\selectfont \textbf{박지민}}}\\
        \vspace{1cm}

        {\fontsize{16pt}{\baselineskip}\selectfont \textbf{박지민의 석사 학위논문을 인준함}}\\
        \vspace{0.2cm}

        {\fontsize{14pt}{\baselineskip}\selectfont \textbf{2025년 2월}}\\
        \vspace{2.5cm}
        
        \makebox[3cm][s]{{\fontsize{16pt}{\baselineskip}\selectfont \textbf{위원장}}}
            \underline{\makebox[7cm][s]{{\fontsize{16pt}{\baselineskip}\selectfont \textbf{\ 전소희\ (인)}}}}\\[2em]
        \makebox[3cm][s]{{\fontsize{16pt}{\baselineskip}\selectfont \textbf{부위원장}}}
            \underline{\makebox[7cm][s]{{\fontsize{16pt}{\baselineskip}\selectfont \textbf{\ 전소희\ (인)}}}}\\[2em]
        \makebox[3cm][s]{{\fontsize{16pt}{\baselineskip}\selectfont \textbf{위원}}}
            \underline{\makebox[7cm][s]{{\fontsize{16pt}{\baselineskip}\selectfont \textbf{\ 전소희\ (인)}}}}\\


    \end{center}
\end{titlepage}




\pagenumbering{roman}           % 목차 등은 소문자 로마자

\begin{center}
    {\LARGE \bfseries 국문초록}
\end{center}


\vspace{1cm}

본 연구는 대한민국의 농업자원경제학 분야에서의 구조적 변화와 정책 대응을 분석한 것이다.
기존 문헌 검토를 통해 주요 이슈를 정리하고, 실증 분석을 통해 최근 10년간의 농업정책 변화가 
자원 배분과 농가소득에 미친 영향을 평가하였다. 주요 결과는 다음과 같다. 첫째, \ldots\ 
둘째, \ldots\ 이러한 결과는 향후 농업정책의 설계에 있어 실질적 함의를 제공할 수 있다.


\noindent
\makebox[1.1cm][s]{{\textbf{주요어}}}
\textbf{:}
\textbf{\abstractkeyword}\\
\makebox[1.1cm][s]{{\textbf{학번}}}
\textbf{:}
\textbf{\studentnumber}\\



\tableofcontents

\newpage

\listoffigures
\listoftables


\newpage

\pagenumbering{arabic}          % 본문은 숫자 페이지
\setcounter{page}{1}


\chapter{제1장 서론}



\section{제 1절 연구 배경}
한국 농업은 기후 변화와 글로벌 시장의 영향으로 다양한 도전에 직면하고 있다.

\section{제 2절 연구 목적}
본 연구는 농업자원경제학 분야에서 정책 변화가 농가 소득에 미친 영향을 분석하고자 한다.

\subsection{1. 세부 목적 1}
정책 변화 전후의 구조 비교

\subsection{2. 세부 목적 2}
실증 분석을 통한 영향 추정

\chapter{제2장 서론}


\section{제 1절 연dd 배경}
한국 농업은 기후 변화와 글로벌 시장의 영향으로 다양한 도전에 직면하고 있다.

\section{제 2절 aaa 목적}
본 연구는 농업자원경제학 분야에서 정책 변화가 농가 소득에 미친 영향을 분석하고자 한다.

\subsection{1. 세부 dfsf적 1}
정책 변화 전후의 구조 비교

\subsection{2. 세부 dfhhjj 2}
실증 분석을 통한 영향 추정
% 추가 장은 자유롭게 추가


\printbibliography[title={참 고 문 헌}]


\begin{center}
  \begin{spacing}{3} % 2.5줄 간격 (2보다 넓음)

    {\LARGE \bfseries Abstract} \par
    {\Huge \bfseries The Economic and} \par
    {\Huge \bfseries Socio-Cultural Effects on} \par
    {\Huge \bfseries Official Developement Assistance} \par
    {\Huge \bfseries on Donor Country} \par

  \end{spacing}
\end{center}

\begin{flushright}
    \begin{spacing}{2}
        {\Large Park Ji Min} \par
        {\Large The Graduate School of Public Administration} \par
        {\Large Seoul National University} \par
     \end{spacing}
\end{flushright}



The purpose of this study is to analyze the economic and cultural
effect of Official Development Assistance(ODA) of Donor Country.
Since joining the OECD Development Assistance Committee (DAC) in
November 2009, Korea has been strengthening its responsibilities and
obligations as a donor country. The scale of Official Development
Assistance (ODA) has more than doubled over the past decade,
increasing from $1.2 billion in 2010 to approximately $3.1 billion in
2023, ranking 14th among all 31 DAC member countries.
ODA fundamentally aims at humanitarian goals, focusing on
poverty alleviation and economic development in the international

\noindent
\makebox[3cm][l]{{\textbf{Keywords}}}
\textbf{:}
\textbf{전소희}\\
\makebox[3cm][l]{{\textbf{Student Number}}}
\textbf{:}
\textbf{2025-20181}\\


\end{document}
