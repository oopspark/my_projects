
\begin{center}
  \begin{spacing}{1.5} % 2.5줄 간격 (2보다 넓음)

    {\LARGE \bfseries Abstract} \par
    {\Huge \bfseries \maintitleen}

  \end{spacing}
\end{center}

\begin{flushright}
    \begin{spacing}{2}
        {\Large \authornameen} \par
        {\Large \affiliationen} \par
        {\Large \univen} \par
     \end{spacing}
\end{flushright}



The purpose of this study is to analyze the economic and cultural
effect of Official Development Assistance(ODA) of Donor Country.
Since joining the OECD Development Assistance Committee (DAC) in
November 2009, Korea has been strengthening its responsibilities and
obligations as a donor country. The scale of Official Development
Assistance (ODA) has more than doubled over the past decade,
increasing from $1.2 billion in 2010 to approximately $3.1 billion in
2023, ranking 14th among all 31 DAC member countries.
ODA fundamentally aims at humanitarian goals, focusing on
poverty alleviation and economic development in the international

\noindent
\makebox[3cm][l]{{\textbf{Keywords}}}
\textbf{:}
\textbf{\abstractkeyworden}\\
\makebox[3cm][l]{{\textbf{Student Number}}}
\textbf{:}
\textbf{\studentnumber}\\
