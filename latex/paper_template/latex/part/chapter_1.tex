\chapter{서 론}



본 연구는 대한민국 농업자원경제학 분야에서 나타나는 구조적 변화와 그에 따른 정책적 대응을 분석하고자 한다.  
한국 농업은 산업화와 도시화, 기후 변화, 고령화, 글로벌 무역환경 변화 등 다양한 요인으로 인해 빠르게 변화하고 있다.  
이러한 외부 충격은 농업 생산 구조뿐만 아니라 농가의 소득 안정성, 자원 배분 효율성에도 큰 영향을 미친다.  
본 연구에서는 최근 10년간의 농업정책 변화가 실제로 어떤 방식으로 농업 구조와 농가 경제에 영향을 미쳤는지를 실증적으로 규명한다.
이 연구는 여러 기존 연구들\cite{hong2020latex,doe2019intro}을 참고하여 수행되었다.
ddssghffgh

---

% 그림 1: 농업 생산량 변화
\begin{figure}[htbp]
  \centering
  \caption{최근 10년간 농업 생산량 변화 추이}\label{fig:chapter1_1}

  \rule{0.7\linewidth}{5cm} % 그림 공간 확보
\end{figure}

---

\section{연구 배경}

한국 농업은 과거 수십 년 간 국가 식량 안보의 핵심 산업으로 기능해왔지만,  
최근에는 기후 변화와 글로벌 공급망의 불안정성, FTA 확대, 농촌 고령화 등으로 위기에 직면하고 있다.  

특히 중소농의 경영 지속 가능성이 악화되고, 농지의 대규모화와 영농법인화가 빠르게 진행되며  
전통적 농업 경제 구조가 재편되는 양상이 뚜렷하다. 또한, 탄소중립과 지속가능성이라는 글로벌 아젠다는  
농업 부문에서도 새로운 기준과 제약을 요구하고 있다.

% 그림 2: 농업 수출 비중 변화
\begin{figure}[htbp]
  \centering
  \rule{0.65\linewidth}{5cm}
  \caption{한국 농업의 수출 비중 변화 추이 (2010–2020)}\label{fig:chapter1_2}
\end{figure}

% 표 1: 농업 GDP 비중 변화
\begin{table}[htbp]
  \centering
  \begin{tabular}{|c|c|c|}
    \hline
    연도 & 농업 GDP (조 원) & 전체 GDP 대비 비중 (\%) \\
    \hline
    2010 & 26.5 & 2.7 \\
    2015 & 23.1 & 1.9 \\
    2020 & 20.4 & 1.3 \\
    \hline
  \end{tabular}
  \caption{2010–2020년 농업 GDP 비중 변화}\label{tab:chapter1_1}
\end{table}



\begin{table} [h]
\caption{\label{tab:stride} Packed view size with different region allocation method. }
\centering
\setlength{\tabcolsep}{10pt}

% tabularx로 \textwidth만큼 꽉 차게
\begin{tabularx}{\textwidth}{c|c|c|Y|Y}
\hline
\rowcolor{gray!30}
& & & \multicolumn{2}{c}{Packed view size} \\\cline{4-5} 
\rowcolor{gray!30}
\multirow{-2}{*}{Sequence name} & \multirow{-2}{*}{Frames} & \multirow{-2}{*}{QP} & No stride & With stride \\

\hline\hline
\csvreader[
  late after line=\\\hline,
  head to column names
]{data/csv/sample.csv}{}{
  \csvcoli & \csvcolii & \csvcoliii & \csvcoliv & \csvcolv
}
\end{tabularx}

\end{table}


\begin{table}
 \begin{tabularx}{\textwidth}{XX}
    foo & bar
\end{tabularx}
\caption{My table}
\end{table}



\begin{tabular}{|c|c|}
\hline
\cellcolor{gray!20} & 셀 A \\
\cline{2-2}
\multirow{-2}{*}{\cellcolor{gray!20}합침} & 셀 B \\
\hline
\end{tabular}




\section{연구 목적}

이 연구의 궁극적인 목적은 한국 농업자원경제 구조에 영향을 미치는 정책 변수들을 식별하고,  
그 영향력을 실증적으로 분석하여 정책의 효율성을 평가하는 것이다.  DID

정책 효과는 단순한 재정 지출이나 보조금 액수가 아니라,  
그 결과로 나타나는 농가의 행태 변화, 소득 변화, 자원 활용 방식 등을 기준으로 평가되어야 한다.  
이를 위해 본 연구는 다음과 같은 세부 목적을 설정한다.

---

\subsection{세부 목적 1}

\textbf{정책 변화 전후의 농업 구조 비교 분석}  

본 절에서는 주요 정책 시행 이전과 이후의 농업 구조를 다양한 지표를 통해 비교한다.  
예컨대, 농가의 평균 경작 면적, 법인 농가 비율, 청년 농업인 유입률, 스마트팜 도입률 등의 변화를 중심으로 분석한다.

% % 표 2: 농가 유형별 구조 변화
% \begin{table}[htbp]
%   \centering
%   \begin{tabular}{|c|c|c|c|}
%     \hline
%     구분 & 2010 & 2020 & 변화율 (\%) \\
%     \hline
%     평균 경작면적 (ha) & 1.1 & 1.8 & +63.6 \\
%     영농법인 비율 (\%) & 3.2 & 8.7 & +171.9 \\
%     청년 농업인 비중 (\%) & 4.8 & 6.2 & +29.2 \\
%     \hline
%   \end{tabular}
%   \caption{2010–2020년 농가 구조 변화 주요 지표}\label{tab:chapter1_2}
% \end{table}

---

\subsection{세부 목적 2}

\textbf{실증 분석을 통한 정책 효과 추정}  

패널 회귀 분석, DID (difference-in-difference), 또는 구조방정식 모형 등을 통해  
주요 정책이 농가소득, 농업생산성, 자원 배분의 효율성에 어떤 영향을 미쳤는지를 추정한다.  

정책 변수로는 쌀 직불제, 친환경 인증제도, 농촌공동체 지원사업 등이 포함된다.

% 그림 3: 정책 참여 농가와 비참여 농가 소득 변화 비교
\begin{figure}[htbp]
  \centering
  \rule{0.65\linewidth}{5cm}
  \caption{정책 참여 여부에 따른 농가 소득 변화}\label{fig:chapter1_3}
\end{figure}

---

\subsection{요약}

서론에서는 한국 농업이 직면한 구조적 변화와 정책 필요성을 다루었으며,  
이후 본 연구가 분석하고자 하는 세부 목적을 ① 구조 비교, ② 실증 분석으로 명확히 구분하였다.  
다음 장에서는 관련 이론과 선행 연구를 검토한다.
