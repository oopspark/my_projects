% 패키지 불러오기
\usepackage{kotex}           % 한글
\newfontfamily\hanjafont{Noto Serif CJK KR}  % 예: 윈도우 내장 중국어 폰트
\usepackage[a4paper,
  left=30mm,
  right=25mm,
  top=30mm,
  bottom=30mm,
  includeheadfoot]{geometry}

\usepackage{multirow}
\usepackage{hhline}
\usepackage{array}
\usepackage{booktabs} 

\usepackage{makecell}
\usepackage[table]{xcolor} % 색상 패키지

\usepackage{indentfirst}


\usepackage{tabularx}
% \usepackage{ltablex} % tabularx의 확장판
% \keepXColumns % 중복 방지

% 열 정의: X열을 포함해서 자동 너비 분배 + 회색 배경 열 정의
% \newcolumntype{C}{>{\columncolor{gray!20}}c}  % 고정폭 회색열
\newcolumntype{Y}{>{\centering \arraybackslash}X}  % 가변폭 열 (tabularx 전용)

\renewcommand{\arraystretch}{1.5}



\usepackage{pgfkeys}
\usepackage{csvsimple}

  % 여백
\usepackage{setspace}     


\usepackage{lmodern}
\usepackage{xparse}
\usepackage{expl3}

\usepackage{fancyhdr}           % 머리말/꼬리말
\usepackage{graphicx}           % 이미지
\usepackage{amsmath,amssymb}    % 수학


\usepackage[
  backend=biber
  ,style=authoryear
  ,language=korean
  ]{biblatex}


\addbibresource{data/references.bib} % .bib 파일 연결


% 폰트 설정 (XeLaTeX인 경우)
% \usepackage{fontspec}
\setmainfont{나눔명조OTF}[
  BoldFont = 나눔명조OTF Bold]


% \renewcommand{\contentsname}{} %  빈칸으로 (아예 안 보이게)
\renewcommand{\chaptername}{제~\thechapter~장} % "제 1 장" 처럼 출력


\usepackage{titlesec}

% 챕터 제목 크기 및 스타일 변경
\titleformat{\chapter}[hang] % display: 별도 줄에 크게
  {\normalfont\Large\bfseries\centering}   % 글꼴: 기본, Huge 크기, 굵게
  {\chaptername }   % 번호 앞에 'Chapter' 붙이기 (영어 기준)
  {20pt}                        % 번호와 제목 사이 간격
  {}                       % 제목 글꼴 (추가 설정)

% 섹션 제목 크기 및 스타일 변경
\renewcommand{\thesection}{\arabic{section}}

\titleformat{\section}
  {\normalfont\Large\bfseries}      % 글꼴
  {제~\thesection~절}                    % 번호 스타일 (여기서 '1.'처럼 점 붙임)
  {1em}                            % 번호와 제목 사이 간격
  {}


% 서브섹션 제목 크기 및 스타일 변경
\renewcommand{\thesubsection}{\arabic{subsection}}

\titleformat{\subsection}
  {\normalfont\large\bfseries} % 기본 크기, 굵게
  {\thesubsection .}
  {1em}
  {}




\renewcommand{\contentsname}{목 차}

% \renewcommand{\figurename}{그림}
% \renewcommand{\tablename}{표}
\renewcommand{\listfigurename}{그림 목차}
\renewcommand{\listtablename}{표 목차}



\usepackage{caption}

\DeclareCaptionLabelFormat{brackets}{[#1~#2]}

\captionsetup{
  labelformat=brackets,
  labelsep=space,
  format=hang,
  font=normalsize
}

\captionsetup[figure]{name=그림}
\captionsetup[table]{name=표}



\makeatletter
\spaceskip=0.7em plus 0.2em minus 0.2em
\makeatother


\titlespacing*{\chapter}{0pt}{0pt}{80pt}


  % 들여쓰기 없음
\setlength{\parskip}{0pt}       % 문단 간격
\linespread{1.5}            % 1.5줄 간격


\usepackage{tocloft}

\renewcommand{\cftchapfont}{\Large\bfseries} % 목차에서 chapter 글씨 크기, 굵게
\renewcommand{\cftsecfont}{\large}            % 목차에서 section 글씨 크기
\renewcommand{\cftsubsecfont}{\normalsize}    % 목차에서 subsection 글씨 크기


\renewcommand{\cftchappagefont}{\Large}  % chapter 페이지 번호 크기
\renewcommand{\cftsecpagefont}{\large}   % section 페이지 번호 크기
\renewcommand{\cftsubsecpagefont}{\normalsize} % subsection 페이지 번호 크


% 목차 항목 간 행간 간격 조절
\setlength{\cftbeforechapskip}{10pt}   % 챕터 항목 위 간격 (행간)
\setlength{\cftbeforesecskip}{5pt}     % 섹션 항목 위 간격
\setlength{\cftbeforesubsecskip}{0pt}  % 서브섹션 항목 위 간격




% \makeatletter
% \patchcmd{\@makechapterhead}{50\p@}{0pt}{}{}
% \patchcmd{\@makeschapterhead}{50\p@}{0pt}{}{}
% \makeatother






\usepackage{etoolbox}

\usepackage{etoc}

% 목차 제목 글꼴/크기 변경 예시 (글꼴 크기와 굵기 조정)
\usepackage{etoolbox}
\makeatletter
\patchcmd{\tableofcontents}
  {\chapter*{\contentsname}}% 원래 코드
  {\chapter*{\fontsize{20}{24}\selectfont\bfseries\centering 목차}}% 중앙정렬 추가
  {}{}
\patchcmd{\listoffigures}
  {\chapter*{\listfigurename}}
  {\chapter*{\fontsize{18}{22}\selectfont\bfseries\centering 그림 목록}}% 중앙정렬 추가
  {}{}
\patchcmd{\listoftables}
  {\chapter*{\listtablename}}
  {\chapter*{\fontsize{18}{22}\selectfont\bfseries\centering 표 목록}}% 중앙정렬 추가
  {}{}
\makeatother